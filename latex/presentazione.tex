\documentclass[handout]{beamer}
\usepackage[italian]{babel}
\usepackage[utf8]{inputenc}
\usepackage[T1]{fontenc}

% TEMA
\usetheme{CambridgeUS}
\usecolortheme{dolphin}
%\setbeamercovered{transparent}

% LOGO
%\pgfdeclareimage[height=2cm]{logo}{logo/unifi}
%\logo{\includegraphics[width=1.5cm]{logo/unifi}}
\titlegraphic{\includegraphics[width=1.5cm]{logo/unifi}}

% DATI
\title[Model checking e sistemi adattivi]{Model checking come supporto \\ per le scelte di sistemi adattivi}
\author[Marco Tinacci]{Marco Tinacci \\ \url{marco.tinacci@gmail.com}}
\institute[Unifi]{Università degli Studi di Firenze \\ \url{www.unifi.it}}
\date{15 Ottobre 2012}

\begin{document}
	% ===============================================================
	% TITOLO
	% ===============================================================
	\begin{frame}
		\maketitle		
	\end{frame}
	% ===============================================================
	% SOMMARIO
	% ===============================================================
	\section*{Sommario}
	\begin{frame}
		\tableofcontents
	\end{frame}
	% ===============================================================
	% SISTEMA ADATTIVO
	% ===============================================================	
	\section{Sistema adattivo}
	\begin{frame}
		\frametitle{Definizione}
		\emph{``In che caso un sistema si può dire adattivo?''}
		\\
		\pause
		Un sistema software è quando il suo \alert{comportamento} dipende da un insieme di \alert{dati di controllo} che possono variare durante l'esecuzione
		\begin{itemize}
			\item esempio 1
			\item esempio 2
			\item esempio 3
		\end{itemize}
	\end{frame}
	
	\begin{frame}
		\frametitle{Obiettivo}
		
	\end{frame}
	% ===============================================================
	% MODEL CHECKING
	% ===============================================================	
	\section{Model checking}

	% ===============================================================
	% LAPSA
	% ===============================================================	
	\section{LAPSA}
%	\frame{lapsa}
	% ===============================================================
	% CASO DI STUDIO
	% ===============================================================	
	\section{Caso di studio}
%	\frame{caso di studio}
	% ===============================================================
	% LAVORI FUTURI
	% ===============================================================	
	\section{Lavori futuri}
%	\frame{lavori futuri}
	
\end{document}