\documentclass{article}

\title{SEAL: a language for self adaptive agents}
\author{Marco Tinacci}
\date{\today}

\usepackage[italian]{babel} % lingua italiana
\usepackage[utf8]{inputenc} % codifica per caratteri con accento
\usepackage{latexsym} % join symbol
\usepackage{amsmath} % x right arrow
\usepackage{amssymb} % triangle equivalence symbol
% MACRO
\newcommand{\Sep}{\quad\mid\quad}
\newcommand{\Space}{\mbox{ }}
\newcommand{\Par}{\Space||_A\Space}

\begin{document}
\maketitle
\tableofcontents
\section{Syntax}

$$
\begin{array}{|rcl@{\ \ }l|}
\hline
S &::=& m(e, \dots, e) \Sep S \Par S & \mbox{(System)} \\
\gamma &::=& \beta [a]\Rightarrow \delta \Sep \gamma,\gamma & \mbox{(Rule)} \\
\beta &::=& \exists \Space x:m \Space . \Space (\beta) \Sep e \Join e \Sep \mathbf{tt} \Sep \beta \vee \beta \Sep \neg\beta & \mbox{(Condition)} \\
\delta &::=& <e> \alpha \Sep \delta \oplus \delta & \mbox{(Distribution)} \\
\alpha &::=& x = e; \Sep \mathbf{no action}; \Sep \alpha \Space \alpha & \mbox{(Action)} \\
\hline
\end{array}
$$

Sono state omesse le descrizioni di alcuni simboli per alleggerire la lettura della sintassi:
\begin{itemize}
	\item $m$: riferimento alla definizione di un modulo,
	\item $A$: insieme di azioni di sincronizzazione,
	\item $e$: espressione,
	\item $\Join$: operatore di confronto,
	\item $x$: dichiarazione di un identificatore,
	\item $a$: label azione.
\end{itemize}

L'insieme dei moduli sarà quindi del tipo
$$ M \subseteq \gamma \times \mathbb{VAR} \times \dots \times \mathbb{VAR} $$

\subsection{Syntactic sugar}
Inseriamo un costrutto tale da poter inserire una condizione di abilitazione sugli elementi del supporto della distribuzione:
$$
\beta [a]\Rightarrow <e,\beta'> \alpha \oplus \delta
\equiv 
\beta \wedge \beta' [a]\Rightarrow <e> \alpha \oplus \delta,
\beta \wedge \neg\beta' [a]\Rightarrow \delta
$$
Introduciamo anche la visuale soggettiva del singolo modulo:
$$
[m]_V \Par m_1 \Par \dots \Par m_n \equiv m \Par V(m_1) \Par \dots \Par V(m_n)
$$
dove $V:M \rightarrow M$ è una funzione che mappa moduli in moduli.

\section{Semantic}
Andiamo a dare un'introduzione informale alla semantica del linguaggio descritto:
\begin{itemize}
	\item \emph{System}: un sistema può essere definito tramite un singolo modulo ($m$ è un riferimento alla definizione di un modulo) o attraverso la composizione parallela di più sistemi su di un insieme di azioni di sincronizzazione $A \subseteq Act$;
	\item \emph{Rule}: un insieme di regole definisce il comportamento di un modulo, se vale la condizione $\beta$ si può passare alla valutazione della distribuzione $\delta$;
	\item \emph{Condition}: descrive una condizione fornendo anche un operatore di quantificatore esistenziale sui moduli;
	\item \emph{Distribution}: descrive una distribuzione probabilistica di azioni $\alpha$, dove ogni azione è accompagnata da un'espressione $e$, che ne descrive il peso, e un'azione $a$;
	\item \emph{Action}: un azione descrive un aggiornamento dello stato, che può consistere nell'assegnamento di una, nessuna, o più variabili.
\end{itemize}

Definiamo la semantica del linguaggio in termini di \emph{Markov Decision Processes}. Alla definizione di ogni modulo sarà assegnata una \emph{MDP} della forma:
$$ (\Sigma,Act,\rightarrow_\rho,\sigma_0) $$
dove 
\begin{itemize}
	\item $\Sigma = \{\sigma | \sigma : \mathbb{VAR} \rightarrow \mathbb{VAL}\}$ è l'insieme degli \emph{stati} rappresentati da funzioni che mappano variabili in valori,
	\item $Act$ l'insieme delle azioni,
	\item $\rightarrow_\rho \subseteq \Sigma \times Act \times Dist(U)$ è la relazione di \emph{avanzamento} di stato,
	\item $\sigma_0 \in \Sigma$ è lo \emph{stato iniziale},
	\item $\rho \subseteq \beta \times Act \times Dist(U)$ è la \emph{struttura statica} del \emph{MDP},
	\item $U = \{u | u : \Sigma \rightarrow \Sigma \}$ è l'insieme delle funzioni \emph{update} di aggiornamento di stato.
\end{itemize}

$$
\begin{array}{cl}
	\displaystyle{\frac{(g,a,d) \in \rho}{\sigma \xrightarrow{a}_\rho d(\sigma)} \Space \sigma \models g} & \mbox{(Update)} \\
\end{array}
$$

$$
\rho_m = \{(g,a,d)\ |\ \gamma_m = g[a] \Rightarrow <e_1> \alpha_1 \oplus \cdots \oplus <e_n> \alpha_n, d=[u_{\alpha_{i 1}}:p_1, \dots, u_{\alpha_{i n}}:p_n]\}
$$

dove $u_\alpha \in U$ è una funzione \emph{update} definita nel seguente modo
$$ 
u_{\alpha}(\sigma) = \left\{
\begin{array}{ll}
	\sigma[eval(e)/x]	& \mbox{se } \alpha = x = e; \\
	\sigma				& \mbox{se } \alpha = \mathbf{noaction}; \\
	u_{\alpha''}(u_{\alpha'}(\sigma))	& \mbox{se } \alpha = \alpha' \alpha'' \\
\end{array}
\right.
$$

Le probabilità sono invece calcolate nel seguente modo
$$ p_i = \frac{eval(e_i)}{\sum_{j=1}^{n}eval(e_j)},i=1,\dots,n $$

Salendo dal livello dei moduli a quello dei sistemi, introduciamo $\Pi \in Dist(S) $ per indicare distribuzioni di sistemi. Il sistema sarà rappresentato dal \emph{MDP} risultante dal parallelo dei \emph{MDP} che lo compongono.

$$
\begin{array}{|cl|}
	\hline
	\displaystyle{\frac{\sigma_m \xrightarrow{a}_{\rho_m} d(\sigma_m)}{S \xrightarrow{a} \Pi}\ m \in S} & \mbox{(Update)} \\[.5cm]
	\displaystyle{\frac{S_1 \xrightarrow{a} \Pi_1 \quad S_2 \xrightarrow{a} \Pi_2}{S_1 \Par S_2 \xrightarrow{a} \Pi_1 \Par \Pi_2}\ a \in A} & \mbox{(Sync)} \\[.5cm]
	\displaystyle{\frac{S_1 \xrightarrow{a} \Pi_1}{S_1 \Par S_2 \xrightarrow{a} \Pi_1 \Par S_2}\ a \not\in A} & \mbox{(Async\ 1)} \\[.5cm]
	\displaystyle{\frac{S_2 \xrightarrow{a} \Pi_2}{S_1 \Par S_2 \xrightarrow{a} S_1 \Par \Pi_2}\ a \not\in A} & \mbox{(Async\ 2)} \\[.5cm]
	\hline
\end{array}
$$

Rimane da definire come si comporta l'operatore di composizione parallela tra un sistema e una distribuzione e tra due distribuzioni.
$$
\Pi_1 \Par S_2 (S) = \left\{
\begin{array}{ll}
	\Pi_1(S_1')	& \mbox{se } S = S_1' \Par S_2 \\
	0			& \mbox{altrimenti} \\
\end{array}
\right.
$$
$$
S_1 \Par \Pi_2 (S) = \left\{
\begin{array}{ll}
	\Pi_2(S_2')	& \mbox{se } S = S_1 \Par S_2' \\
	0			& \mbox{altrimenti} \\
\end{array}
\right.
$$
$$
\Pi_1 \Par \Pi_2 (S) = \left\{
\begin{array}{ll}
	\Pi_1(S_1)\cdot\Pi_2(S_2)	& \mbox{se } S = S_1 \Par S_2 \\
	0			& \mbox{altrimenti} \\
\end{array}
\right.
$$


\section{Examples}
Esempio di un modulo di robot che esegue una \emph{random walk} su una griglia escludedo dalla scelta probabilistica le direzioni adiacenti occupate:
$$
\begin{array}{rcl}
m_1 () & \triangleq & \mathbf{tt} [step]\Rightarrow \\ 
	& & <1, \neg \Space \exists \Space m_1 \Space v: v.x=x \wedge v.y=y+1> y=y+1; \oplus \\
	& & <1, \neg \Space \exists \Space m_1 \Space v: v.x=x \wedge v.y=y-1> y=y-1; \oplus \\ 
	& & <1, \neg \Space \exists \Space m_1 \Space v: v.x=x+1 \wedge v.y=y> x=x+1; \oplus \\ 
	& & <1, \neg \Space \exists \Space m_1 \Space v: v.x=x-1 \wedge v.y=y> x=x-1; \oplus \\ 
	& & <1, \mathbf{tt}> \mathbf{noaction};\\ 
\end{array}
$$
Esempio di un modulo di robot analogo al precedente con la differenza che la scelta della mossa viene fatta in modo nondeterministico:
$$
\begin{array}{rcllcl}
m_2 () & \triangleq & \neg \Space \exists \Space m_1 \Space v: (v.x=x \wedge v.y=y+1) &[north]&\Rightarrow& <1> y=y+1; \\ 
	& & \neg \Space \exists \Space m_1 \Space v: (v.x=x \wedge v.y=y-1) &[south]&\Rightarrow& <1> y=y-1; \\ 
	& & \neg \Space \exists \Space m_1 \Space v: (v.x=x+1 \wedge v.y=y) &[east]&\Rightarrow& <1> x=x+1; \\ 
	& & \neg \Space \exists \Space m_1 \Space v: (v.x=x-1 \wedge v.y=y) &[west]&\Rightarrow& <1> x=x-1; \\ 
	& & \mathbf{tt} &[stay]&\Rightarrow& <1> \mathbf{noaction}; \\ 
\end{array}
$$

\end{document}
