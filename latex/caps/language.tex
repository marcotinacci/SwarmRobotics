%!TEX root = ../main.tex
\section{Syntax}
Per descrivere la sintassi \ac{seal} sarà utilizzato il formalismo \emph{EBNF} indicando con le parole in corsivo i simboli non terminali e con quelle in neretto e quelle in stampatello i non terminali. Le parole in neretto sono \emph{keyword} del linguaggio mentre quelle in stampatello descrivono dei valori arbitrari su insiemi come nomi di variabili o costanti numeriche.

\begin{table} % sintassi programma SEAL
$$
\begin{array}{|rcl|}
\hline
\mathit{program} &::=& \mathbf{actions} \{ \mathit{actions} \} \\
	&& \mathbf{subject} \Space \mathit{module} \\
	&& \mathit{targets} \\
	&& \mathit{environment} \\
	&& \mathbf{ranges} \{ \mathit{ranges} \}
	\\[.3cm]
\mathit{actions} &::=& \x{action}\textrm{-}\x{label} \Sep \mathit{actions} , \mathit{actions}
	\\[.3cm]
\mathit{targets} &::=& \dots
	\\[.3cm]	
\mathit{environment} &::=& \mathit{module} \Sep \mathit{environment} \Space \mathit{environment} 
	\\
\hline
\end{array}
$$
\label{tab:sealProg} % sintassi programma SEAL
\caption{Sintassi di un programma \ac{seal}}
\end{table}

Il non terminale \emph{program} è il simbolo iniziale della grammatica e nella sua struttura racchiude la dichiarazione dell'insieme di azioni considerato, il modulo che descrive il comportamento del soggetto, gli obiettivi del soggetto, i moduli che possono essere usati per descrivere l'ambiente e i dati necessari alla discretizzazione delle variabili.

In tabella \ref{tab:sealProg} viene descritta anche la sintassi dell'insieme delle azioni, degli obiettivi e dell'ambiente. Le azioni e l'ambiente sono una semplice lista, rispettivamente, di \emph{label} e di \emph{module}. La lista di moduli di ambiente non esprime direttamente quali moduli sono considerati presenti nell'ambiente ma quelli che possono esserci inseriti.

% TODO sintassi obiettivo

\begin{table} % sintassi modulo SEAL
$$
\begin{array}{|rcl|}
\hline
\mathit{module} &::=& \mathbf{module} \Space \x{module}\textrm{-}\x{name} \{ variables \Space rules \}
	\\[.3cm]
\mathit{variables} &::=& \x{type} \Space \x{id} = \mathit{expression}; \Sep \mathit{variables} \Space \mathit{variables}
	\\[.3cm]
\mathit{rules} &::=& \mathit{condition} [ \x{action}\textrm{-}\x{label}]\Rightarrow \mathit{distribution} \Sep \mathit{rules},\mathit{rule}
	\\[.3cm]
\mathit{condition} &::=& \mathbf{E} \Space \x{id}:\x{module}\textrm{-}\x{name} \Space . \Space \mathit{condition} \Sep \mathit{expression} \Join \mathit{expression} \\
	& \Sep & \mathbf{true} \Sep \mathit{condition} \Space \mathbf{or} \Space \mathit{condition} \Sep \x{!} \Space \mathit{condition}
	\\[.3cm]
\mathit{distribution} &::=& <\mathit{expression}> \mathit{update}; \Sep \mathit{distribution} \Space \# \Space \mathit{distribution}
	\\[.3cm]
\mathit{update} &::=& \x{id} = \mathit{expression} \Sep \mathbf{noaction} \Sep \mathit{update} \Space ; \Space \mathit{update} \\
	& \Sep & \mathbf{env.add} \Space \mathit{system} \Sep \mathbf{env.remove} \Space \x{id}:\x{module}\textrm{-}\x{name} \Space . \Space \mathit{condition} 
	\\
\hline
\end{array}
$$
\label{tab:sealMod}
\caption{Sintassi di un modulo \ac{seal}}
\end{table}

La definizione di un modulo (tabella \ref{tab:sealMod}) consiste nell'attribuzione di un nome che lo identifica, un insieme di variabili che ne descrive lo stato, e un insieme di regole che ne descrive l'avanzamento. Le variabili vengono dichiarate all'inizio del modulo specificandone il tipo, il nome e il valore iniziale attribuito. Una regola, invece, è composta da una condizione di guarda che la abilita, un'azione da eseguire, e una distribuzione di possibli funzioni di \emph{update}, ognuna delle quali ha associata una probabilità di essere eseguita. Le probabilità della distribuzione vengono determinate dalla normalizzazione dei pesi associati alle funzioni. La funzione \emph{update} è una sequenza di azioni di aggiornamento di stato o di visione dell'ambiente.

\begin{itemize}
	\item $A$
	\item $\Join$
	\item $\x{id}$
	\item $\x{module}\textrm{-}\x{name}$
	\item $\x{action}\textrm{-}\x{label}$
	\item $\x{type}$
	\item $\x{value}$
\end{itemize}

\begin{table} % sintassi completa SEAL
$$
\begin{array}{|rcl|}
\hline
\mathit{program} &::=& \mathbf{actions} \{ \mathit{actions} \} \\
	&& \mathbf{subject} \Space \mathit{module} \\
	&& \mathit{targets} \\
	&& \mathit{environment} \\
	&& \mathbf{ranges} \{ \mathit{ranges} \}
	\\[.3cm]
\mathit{targets} &::=& \dots
	\\[.3cm]	
\mathit{environment} &::=& \mathit{module} \Sep \mathit{environment} \Space \mathit{environment}
	\\[.3cm]
\mathit{actions} &::=& \x{action}\textrm{-}\x{label} \Sep \mathit{actions} , \mathit{actions}
	\\[.3cm]
\mathit{ranges} &::=& \x{module}\textrm{-}\x{name} . \x{id} \Space \mathbf{in} ( \x{value} , \x{value}, \x{value} )
	\\[.3cm]
\mathit{module} &::=& \mathbf{module} \Space \x{module}\textrm{-}\x{name} \{ variables \Space rules \}
	\\[.3cm]
\mathit{variables} &::=& \x{type} \Space \x{id} = \mathit{expression}; \Sep \mathit{variables} \Space \mathit{variables}
	\\[.3cm]
\mathit{expression} &::=& \dots
	\\[.3cm]
\mathit{system} &::=& \x{module}\textrm{-}\x{name}(\mathit{params}) \Sep \mathit{system} \Par{A} \mathit{system}
	\\[.3cm]
\mathit{params} &::=& \mathbf{void} \Sep \mathit{expression} \Sep \mathit{params}, \mathit{params}
	\\[.3cm]
\mathit{rules} &::=& \mathit{condition} [ \x{action}\textrm{-}\x{label}]\Rightarrow \mathit{distribution} \Sep \mathit{rules},\mathit{rule}
	\\[.3cm]
\mathit{condition} &::=& \mathbf{E} \Space \x{id}:\x{module}\textrm{-}\x{name} \Space . \Space \mathit{condition} \Sep \mathit{expression} \Join \mathit{expression} \\
	& \Sep & \mathbf{true} \Sep \mathit{condition} \Space \mathbf{or} \Space \mathit{condition} \Sep \x{!} \Space \mathit{condition}
	\\[.3cm]
\mathit{distribution} &::=& <\mathit{expression}> \mathit{update}; \Sep \mathit{distribution} \Space \# \Space \mathit{distribution}
	\\[.3cm]
\mathit{update} &::=& \x{id} = \mathit{expression} \Sep \mathbf{noaction} \Sep \mathit{update} \Space ; \Space \mathit{update} \\
	& \Sep & \mathbf{env.add} \Space \mathit{system} \Sep \mathbf{env.remove} \Space \x{id}:\x{module}\textrm{-}\x{name} \Space . \Space \mathit{condition} 
	\\[.2cm]
\hline
\end{array}
$$
\label{tab:sealsyntax}
\caption{Sintassi \ac{seal} completa}
\end{table}

L'insieme dei moduli sarà quindi del tipo
$$ M \subseteq \gamma \times \mathbb{VAR} \times \dots \times \mathbb{VAR} $$

\subsection{Syntactic sugar}
Inseriamo un costrutto tale da poter inserire una condizione di abilitazione sugli elementi del supporto della distribuzione:
$$
\beta [a]\Rightarrow <e,\beta'> \alpha \oplus \delta
\equiv 
\beta \wedge \beta' [a]\Rightarrow <e> \alpha \oplus \delta,
\beta \wedge \neg\beta' [a]\Rightarrow \delta
$$

\section{Semantic}
Andiamo a dare un'introduzione informale alla semantica del linguaggio descritto:
\begin{itemize}
	\item \emph{System}: un sistema può essere definito tramite un singolo modulo ($m$ è un riferimento alla definizione di un modulo) o attraverso la composizione parallela di più sistemi su di un insieme di azioni di sincronizzazione $A \subseteq Act$;
	\item \emph{Rule}: un insieme di regole definisce il comportamento di un modulo, se vale la condizione $\beta$ si può passare alla valutazione della distribuzione $\delta$;
	\item \emph{Condition}: descrive una condizione fornendo anche un operatore di quantificatore esistenziale sui moduli;
	\item \emph{Distribution}: descrive una distribuzione probabilistica di azioni $\alpha$, dove ogni azione è accompagnata da un'espressione $e$, che ne descrive il peso, e un'azione $a$;
	\item \emph{Action}: un azione descrive un aggiornamento dello stato, che può consistere nell'assegnamento di una, nessuna, o più variabili.
\end{itemize}

Definiamo la semantica del linguaggio in termini di \ac{mdp}. Alla definizione di ogni modulo sarà assegnata una \ac{mdp} della forma:
$$ (\Sigma,Act,\rightarrow_\rho,\sigma_0) $$
dove 
\begin{itemize}
	\item $\Sigma = \{\sigma | \sigma : \mathbb{VAR} \rightarrow \mathbb{VAL}\}$ è l'insieme degli \emph{stati} rappresentati da funzioni che mappano variabili in valori,
	\item $Act$ l'insieme delle azioni,
	\item $\rightarrow_\rho \subseteq \Sigma \times Act \times Dist(U)$ è la relazione di \emph{avanzamento} di stato,
	\item $\sigma_0 \in \Sigma$ è lo \emph{stato iniziale},
	\item $\rho \subseteq \beta \times Act \times Dist(U)$ è la \emph{struttura statica} del \ac{mdp},
	\item $U = \{u | u : \Sigma \rightarrow \Sigma \}$ è l'insieme delle funzioni \emph{update} di aggiornamento di stato.
\end{itemize}

$$
\begin{array}{cl}
	\displaystyle{\frac{(g,a,d) \in \rho}{\sigma \xrightarrow{a}_\rho d(\sigma)} \Space \sigma \models g} & \mbox{(Update)} \\
\end{array}
$$

$$
\rho_m = \{(g,a,d)\ |\ \gamma_m = g[a] \Rightarrow <e_1> \alpha_1 \oplus \cdots \oplus <e_n> \alpha_n, d=[u_{\alpha_{i 1}}:p_1, \dots, u_{\alpha_{i n}}:p_n]\}
$$

dove $u_\alpha \in U$ è una funzione \emph{update} definita nel seguente modo
$$ 
u_{\alpha}(\sigma) = \left\{
\begin{array}{ll}
	\sigma[eval(e)/x]	& \mbox{se } \alpha = x = e; \\
	\sigma				& \mbox{se } \alpha = \mathbf{noaction}; \\
	u_{\alpha''}(u_{\alpha'}(\sigma))	& \mbox{se } \alpha = \alpha' \alpha'' \\
\end{array}
\right.
$$

Le probabilità sono invece calcolate nel seguente modo
$$ p_i = \frac{eval(e_i)}{\sum_{j=1}^{n}eval(e_j)},i=1,\dots,n $$

Salendo dal livello dei moduli a quello dei sistemi, introduciamo $\Pi \in Dist(S) $ per indicare distribuzioni di sistemi. Il sistema sarà rappresentato dalla \ac{mdp} risultante dal parallelo di quelle che lo compongono.

$$
\begin{array}{|cl|}
	\hline
	\displaystyle{\frac{\sigma_m \xrightarrow{a}_{\rho_m} d(\sigma_m)}{S \xrightarrow{a} \Pi}\ m \in S} & \mbox{(Update)} \\[.5cm]
	\displaystyle{\frac{S_1 \xrightarrow{a} \Pi_1 \quad S_2 \xrightarrow{a} \Pi_2}{S_1 \Par{A} S_2 \xrightarrow{a} \Pi_1 \Par{A} \Pi_2}\ a \in A} & \mbox{(Sync)} \\[.5cm]
	\displaystyle{\frac{S_1 \xrightarrow{a} \Pi_1}{S_1 \Par{A} S_2 \xrightarrow{a} \Pi_1 \Par{A} S_2}\ a \not\in A} & \mbox{(Async\ 1)} \\[.5cm]
	\displaystyle{\frac{S_2 \xrightarrow{a} \Pi_2}{S_1 \Par{A} S_2 \xrightarrow{a} S_1 \Par{A} \Pi_2}\ a \not\in A} & \mbox{(Async\ 2)} \\[.5cm]
	\hline
\end{array}
$$

Rimane da definire come si comporta l'operatore di composizione parallela tra un sistema e una distribuzione e tra due distribuzioni.
$$
\Pi_1 \Par{A} S_2 (S) = \left\{
\begin{array}{ll}
	\Pi_1(S_1')	& \mbox{se } S = S_1' \Par{A} S_2 \\
	0			& \mbox{altrimenti} \\
\end{array}
\right.
$$
$$
S_1 \Par{A} \Pi_2 (S) = \left\{
\begin{array}{ll}
	\Pi_2(S_2')	& \mbox{se } S = S_1 \Par{A} S_2' \\
	0			& \mbox{altrimenti} \\
\end{array}
\right.
$$
$$
\Pi_1 \Par{A} \Pi_2 (S) = \left\{
\begin{array}{ll}
	\Pi_1(S_1)\cdot\Pi_2(S_2)	& \mbox{se } S = S_1 \Par{A} S_2 \\
	0			& \mbox{altrimenti} \\
\end{array}
\right.
$$


\section{Examples}
Esempio di un modulo di robot che esegue una \emph{random walk} su una griglia escludedo dalla scelta probabilistica le direzioni adiacenti occupate:
$$
\begin{array}{rcl}
m_1 () & \triangleq & \mathbf{true} [step]\Rightarrow \\ 
	& & <1, \neg \Space \exists \Space m_1 \Space v: v.x=x \Space \mathbf{and} \Space v.y=y+1> y=y+1; \# \\
	& & <1, \neg \Space \exists \Space m_1 \Space v: v.x=x \Space \mathbf{and} \Space v.y=y-1> y=y-1; \# \\ 
	& & <1, \neg \Space \exists \Space m_1 \Space v: v.x=x+1 \Space \mathbf{and} \Space v.y=y> x=x+1; \# \\ 
	& & <1, \neg \Space \exists \Space m_1 \Space v: v.x=x-1 \Space \mathbf{and} \Space v.y=y> x=x-1; \# \\ 
	& & <1, \mathbf{true}> \mathbf{noaction};\\ 
\end{array}
$$
Esempio di un modulo di robot analogo al precedente con la differenza che la scelta della mossa viene fatta in modo nondeterministico:
$$
\begin{array}{rcllcl}
m_2 () & \triangleq & \neg \Space E \Space m_1 : v . (v.x=x \wedge v.y=y+1) &[north]&\Rightarrow& <1> y=y+1; \\ 
	& & \neg \Space E \Space m_1 : v . (v.x=x \wedge v.y=y-1) &[south]&\Rightarrow& <1> y=y-1; \\ 
	& & \neg \Space E \Space m_1 : v . (v.x=x+1 \wedge v.y=y) &[east]&\Rightarrow& <1> x=x+1; \\ 
	& & \neg \Space E \Space m_1 : v . (v.x=x-1 \wedge v.y=y) &[west]&\Rightarrow& <1> x=x-1; \\ 
	& & \mathbf{true} &[stay]&\Rightarrow& <1> \mathbf{noaction}; \\ 
\end{array}
$$

\section{Code translation}
\subsection{Prism}
La traduzione del sorgente \ac{seal} in codice \prism{} necessita di alcuni dati aggiuntivi riguardo le variabili dei moduli. Dato che \prism{} lavora con uno spazio degli stati finito è necessario aggiungere informazioni che permettano di trattare le variabili intere e reali. Per le variabili intere sarà sufficiente specificare il range, mentre per le variabili reali dovrà essere specificata anche l'ampiezza dell'intervallo di discretizzazione.

La traduzione delle variabili viene descritta in tabella \ref{tab:sealtoprism}. Con \texttt{a}, \texttt{b} e \texttt{delta} rappresentiamo costanti intere, con \texttt{e} un'espressione \ac{seal} e con \texttt{e'} la rispettiva traduzione in \prism{}. I dati necessari alla discretizzazione vengono attualmente forniti direttamente nel file \prism{} nella sezione \emph{ranges}.
\begin{table}[htbp!]
\centering
\begin{tabular}{|l|l|l|}
	\hline
	\ac{seal} & \emph{Input file} &\prism{} \\
	\hline
	\texttt{module m\{} & - & \texttt{x:bool init e';} \\
	\texttt{...} & & \\
	\texttt{bool x = e;}  & & \\
	\texttt{...} & & \\
	\texttt{\}} & & \\
	\hline
	\texttt{module m \{} & \texttt{m.y in (a,b);} & \texttt{y:[a..b] init e';} \\
	\texttt{...} & & \\
	\texttt{int y = e;} & & \\
	\texttt{...} & & \\
	\texttt{\}} & & \\
	\hline
	\texttt{module m \{} & \texttt{m.z in (a,b,delta);} & \texttt{z:[0..floor((a-b)/delta)]} \\
	\texttt{...} & & \texttt{init ceil((e'-a)/delta);} \\
	\texttt{float z = e;} & & \\
	\texttt{...} & & \\
	\texttt{\}} & & \\
	\hline
	\texttt{z = e} & \texttt{m.z in (a,b,delta);} & \texttt{z' = ceil((e'-a)/delta)} \\
	\hline
\end{tabular}
\caption{Traduzione da \ac{seal} a \prism{}}
\label{tab:sealtoprism}
\end{table}
