%*******************************************************
% Abstract
%*******************************************************
\renewcommand{\abstractname}{Sommario}
\pdfbookmark[1]{Abstract}{Sommario}
\begingroup
\let\clearpage\relax
\let\cleardoublepage\relax
\let\cleardoublepage\relax

\chapter*{Sommario}
Il problema affrontato è la ricerca di strategie per agenti adattivi, agenti che devono prendere decisioni che portino al raggiungimento dei loro obiettivi avendo a disposizione una conoscenza parziale o nulla degli elementi che compongono l'ambiente in cui agiscono. L'idea di base è quella di utilizzare un model checker su un modello di ambiente costruito su delle ipotesi e verificando una formula che formalizza l'obiettivo. Si fa quindi un utilizzo non convenzionale del model checker che viene impiegato per fare previsioni anzichÈ verifiche. Presentiamo poi \acl{lapsa}, un linguaggio specifico per modellare agenti adattivi, e la rispettiva implementazione in \xtext{}. Il model checker utilizzato nell'implementazione proposta è \prism{}. Viene infine mostrato il funzionamento in un caso di studio dove è necessario fornire a un agente mobile uno scheduler che minimizzi gli scontri con altri agenti presenti nelle vicinanze.
\endgroup			

\vfill